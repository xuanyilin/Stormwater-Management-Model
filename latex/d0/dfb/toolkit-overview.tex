The Programmer\textquotesingle{}s Toolkit application programming interface (A\+PI) is an extension of the S\+W\+MM simulation package. Through the A\+PI, a developer can programmatically talk to S\+W\+MM before, during, and after a simulation. The A\+PI exposes S\+W\+MM\textquotesingle{}s data model which enables new possibility. When a hydraulic network is initialized, a network is most generally comprised of nodes, link, and runoff surfaces. Nodes and Links are normally manholes and conduits, and runoff surfaces are a hydrologic representation of rainfall catchment surfaces that send flow into the hydraulic network.

The Toolkit A\+PI provides a series of functions that allow programmers to customize the use of S\+W\+MM\textquotesingle{}s hydraulic and water quality solution engine to their own applications. Before using the Toolkit one should become familiar with the way that S\+W\+MM represents the hydrologic surfaces and hydraulic network and the design and operating information it requires to perform a simulation. This information can be obtained from reading S\+W\+MM\textquotesingle{}s Users Manuals \href{https://www.epa.gov/water-research/storm-water-management-model-swmm}{\tt U\+S\+E\+PA Link}.

A typical usage of the Toolkit functions to analyze a stormwater network might look as follows\+:


\begin{DoxyEnumerate}
\item Use the \hyperlink{swmm5_8h_a2da4baa38e451ef2a978f1366898948e}{swmm\+\_\+open} function to open the Toolkit system, along with a S\+W\+MM \mbox{[}Input file\mbox{]}(Input-\/\+File).
\item Use the {\ttfamily swmm\+\_\+setxxx\+Param} to update parameters before beginning a simulation
\item Use the \hyperlink{swmm5_8h_a06eb7b28597a24f892edfc303e12b2d7}{swmm\+\_\+start} function to start the simulation.
\item Iterator over the simlation using \hyperlink{swmm5_8h_a20624449e77d94d71c0e61d7f58558b5}{swmm\+\_\+step} function and exit the loop when a 0 is returned. While the simlation is running, use the {\ttfamily swmm\+\_\+getxxx\+Result} or swmm\+\_\+getxxx\+Stats` to read results.
\item Use the \hyperlink{swmm5_8h_a8e69e10ab4948522041deb92fc4b43c0}{swmm\+\_\+end} function to end a simulation and save the results.
\item Use the \hyperlink{swmm5_8h_a76dac918844d9551a1b39432284b70ef}{swmm\+\_\+close} function to close the simulation and free the memory allocated from the simulation.
\end{DoxyEnumerate}
\begin{DoxyItemize}
\item \hyperlink{how-to-use}{How to Use the Toolkit} 
\end{DoxyItemize}\hypertarget{how-to-use}{}\section{How to Use the Toolkit}\label{how-to-use}
Full Function collection can be found \hyperlink{toolkit_a_p_i_8h}{toolkit\+A\+P\+I.\+h}


\begin{DoxyCode}
\textcolor{preprocessor}{#include <stdio.h>}
\textcolor{preprocessor}{#include <stdlib.h>}
\textcolor{preprocessor}{#include <string.h>}

\textcolor{preprocessor}{#include "\hyperlink{swmm5_8h}{swmm5.h}"}
\textcolor{preprocessor}{#include "\hyperlink{toolkit_a_p_i_8h}{toolkitAPI.h}"}

\textcolor{keywordtype}{long} newHour, oldHour = 0;
\textcolor{keywordtype}{long} theDay, theHour;
\textcolor{keywordtype}{double} elapsedTime = 0.0;

\textcolor{keywordtype}{char} *inputFile;
\textcolor{keywordtype}{char} *reportFile;
\textcolor{keywordtype}{char} *binaryFile;

inputFile = \textcolor{stringliteral}{"<path2>/inputfile.inp"};
reportFile = \textcolor{stringliteral}{"<path2>/reportfile.rpt"};
binaryFile = \textcolor{stringliteral}{"<path2>/outputfile.out"};

\textcolor{comment}{// Open the files & read input data}
ErrorCode = \hyperlink{swmm5_8h_a2da4baa38e451ef2a978f1366898948e}{swmm\_open}(inputFile, reportFile, binaryFile);

\textcolor{comment}{// Run the simulation if input data OK}
\textcolor{keywordflow}{if} ( !ErrorCode )
\{
    \textcolor{keywordtype}{int} ndType;
    \textcolor{keywordtype}{double} depth = 0;

    \hyperlink{group___network_info_gae66a64aee207928c902cb399222cf418}{swmm\_getNodeType}(0, &ndType);

    printf(\textcolor{stringliteral}{"Node Type: %d"}, ndType); \(\backslash\)\(\backslash\) Print node type (See \hyperlink{toolkit_a_p_i_8h_a10b2a8f7c7f2a84ec03332d6cf963cf1}{SM\_NodeType})

    \textcolor{comment}{// Initialize values and Start the Simulation}
    ErrorCode = \hyperlink{swmm5_8h_a06eb7b28597a24f892edfc303e12b2d7}{swmm\_start}(TRUE);

    \textcolor{comment}{// Execute each time step until elapsed time is re-set to 0}
    \textcolor{keywordflow}{if} ( !ErrorCode )
    \{
        \textcolor{keywordflow}{do}
        \{
            ErrorCode = \hyperlink{swmm5_8h_a20624449e77d94d71c0e61d7f58558b5}{swmm\_step}(&elapsedTime);

            \hyperlink{group__tkfuncs_gacef44d121a8f9a646f4e6f108b494373}{swmm\_getNodeResult}(0, \hyperlink{toolkit_a_p_i_8h_aa5c50a52f0aa8cb1bcbb5130e8fa64e2a36c502c185c63b612ddc65ad469d027f}{SM\_NODEDEPTH}, &depth) \(\backslash\)\(\backslash\) Stream Results!

            printf(\textcolor{stringliteral}{"Node Depth %lf"}, depth); \(\backslash\)\(\backslash\) Print node result (See 
      \hyperlink{toolkit_a_p_i_8h_aa5c50a52f0aa8cb1bcbb5130e8fa64e2}{SM\_NodeResult})

        \} \textcolor{keywordflow}{while} ( elapsedTime > 0.0 && !ErrorCode );
    \}
    \textcolor{comment}{// Clean up}
    ErrorCode = \hyperlink{swmm5_8h_a8e69e10ab4948522041deb92fc4b43c0}{swmm\_end}();
\}

\textcolor{comment}{// Get Stats for Node}
\hyperlink{struct_s_m___node_stats}{SM\_NodeStats}* Node0Stats;
\hyperlink{group__tkfuncs_ga0d52d8f1b900a0ef3d3df309c5df21f8}{swmm\_getNodeStats}(0, &Node0Stats);

printf(\textcolor{stringliteral}{"Node Max Depth %lf"}, Node0Stats->avgDepth);

\textcolor{comment}{// Report results}
\hyperlink{swmm5_8h_aa19010a1a938610a22ef3e256a1e8604}{swmm\_report}();

\textcolor{comment}{// Close the system}
\hyperlink{swmm5_8h_a76dac918844d9551a1b39432284b70ef}{swmm\_close}();
\end{DoxyCode}
 